% Options for packages loaded elsewhere
\PassOptionsToPackage{unicode}{hyperref}
\PassOptionsToPackage{hyphens}{url}
%
\documentclass[
]{article}
\usepackage{amsmath,amssymb}
\usepackage{lmodern}
\usepackage{iftex}
\ifPDFTeX
  \usepackage[T1]{fontenc}
  \usepackage[utf8]{inputenc}
  \usepackage{textcomp} % provide euro and other symbols
\else % if luatex or xetex
  \usepackage{unicode-math}
  \defaultfontfeatures{Scale=MatchLowercase}
  \defaultfontfeatures[\rmfamily]{Ligatures=TeX,Scale=1}
\fi
% Use upquote if available, for straight quotes in verbatim environments
\IfFileExists{upquote.sty}{\usepackage{upquote}}{}
\IfFileExists{microtype.sty}{% use microtype if available
  \usepackage[]{microtype}
  \UseMicrotypeSet[protrusion]{basicmath} % disable protrusion for tt fonts
}{}
\makeatletter
\@ifundefined{KOMAClassName}{% if non-KOMA class
  \IfFileExists{parskip.sty}{%
    \usepackage{parskip}
  }{% else
    \setlength{\parindent}{0pt}
    \setlength{\parskip}{6pt plus 2pt minus 1pt}}
}{% if KOMA class
  \KOMAoptions{parskip=half}}
\makeatother
\usepackage{xcolor}
\IfFileExists{xurl.sty}{\usepackage{xurl}}{} % add URL line breaks if available
\IfFileExists{bookmark.sty}{\usepackage{bookmark}}{\usepackage{hyperref}}
\hypersetup{
  pdftitle={Prática de Computador I: Método Científico},
  pdfauthor={Marco A. R. Mello},
  hidelinks,
  pdfcreator={LaTeX via pandoc}}
\urlstyle{same} % disable monospaced font for URLs
\usepackage[margin=1in]{geometry}
\usepackage{graphicx}
\makeatletter
\def\maxwidth{\ifdim\Gin@nat@width>\linewidth\linewidth\else\Gin@nat@width\fi}
\def\maxheight{\ifdim\Gin@nat@height>\textheight\textheight\else\Gin@nat@height\fi}
\makeatother
% Scale images if necessary, so that they will not overflow the page
% margins by default, and it is still possible to overwrite the defaults
% using explicit options in \includegraphics[width, height, ...]{}
\setkeys{Gin}{width=\maxwidth,height=\maxheight,keepaspectratio}
% Set default figure placement to htbp
\makeatletter
\def\fps@figure{htbp}
\makeatother
\setlength{\emergencystretch}{3em} % prevent overfull lines
\providecommand{\tightlist}{%
  \setlength{\itemsep}{0pt}\setlength{\parskip}{0pt}}
\setcounter{secnumdepth}{-\maxdimen} % remove section numbering
\ifLuaTeX
  \usepackage{selnolig}  % disable illegal ligatures
\fi

\title{Prática de Computador I: Método Científico}
\author{Marco A. R. Mello}
\date{}

\begin{document}
\maketitle

Universidade de São Paulo

Instituto de Biociências

Departamento de Ecologia

\href{https://uspdigital.usp.br/jupiterweb/obterDisciplina?sgldis=BIE0315\&verdis=4}{Tópicos
Avançados em Ecologia de Animais (BIE0315)}

Profs. José Carlos Motta Jr.~\& Marco A. R. Mello

Artigo de referência: \href{https://doi.org/10.1038/444701a}{Muchhala
(2006, Nature)}

Agradecimentos: Renata Muylaert escreveu o post sobre como plotar mapais
mentais usando LaTeX.

\href{https://github.com/marmello77/EcoAnimal\#readme}{README}

\hypertarget{instruuxe7uxf5es}{%
\subsection{Instruções}\label{instruuxe7uxf5es}}

Primeiro, leia o tutorial desta prática, disponível aqui em formato PDF
e também no moodle da disciplina.

Aqui apresentamos uma solução alternativa para fazer um mapa mental por
programação. Ela se baseia na linguagem LaTeX, programada na plataforma
Overleaf. Trata-se de uma plataforma online e gratuita. Basta se
cadastrar para poder usá-la.

Para experimentar essa solução, leia este
\href{https://marcoarmello.wordpress.com/2022/04/13/como-programar-um-mapa-mental/}{post
no blog Sobrevivendo na Ciência} e siga as instruções dadas.

\hypertarget{para-saber-mais}{%
\subsection{Para saber mais}\label{para-saber-mais}}

\href{https://marcoarmello.wordpress.com/2015/04/15/mapasmentais/}{Mapa
mental: organizando suas ideias}

\href{https://marcoarmello.wordpress.com/2022/04/13/como-programar-um-mapa-mental/}{Como
programar um mapa mental}

\end{document}
